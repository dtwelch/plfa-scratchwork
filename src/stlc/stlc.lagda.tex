\documentclass[runningheads]{llncs}
\usepackage[margin=1.85cm, paperwidth=6.1in, paperheight=9in, bottom=2.5cm]{geometry}

\usepackage{mathptmx} % times font instead of cm-modern
\usepackage{textgreek}
\usepackage{ucs}
\usepackage{agda}
\usepackage[utf8x]{inputenc}

%\DeclareUnicodeCharacter{948}{\ensuremath{\delta}}
%\DeclareUnicodeCharacter{955}{\ensuremath{\lambda}}

\usepackage{xcolor}
\usepackage{graphicx}
\pagecolor{white}

\definecolor{body}{RGB}{30, 30, 30}
\color{body}

\begin{document}

\title{Notes on formalizing the STLC}
\author{D. Welch}
\institute{}
\maketitle

\begin{abstract}
Just some notes trying out ``\LaTeX \ literate Agda mode'' as I work through 
this next chapter on the lambda calculus (eventually denotational semantics for
the untyped version of the calculus).
\end{abstract}

\section{Intro to the Lambda Calculus} 

The lambda calculus, first published by Alonzo Church in 1932 is a tiny language 
with only three syntactic constructs: variables, abstraction, and (function) 
application. 

The \textit{simply typed lambda calculus} (STLC) is a variant of the lambda 
calculus published in 1940 that adds static typing to the o.g. 1932 untyped 
lambda calculus. It has the three types of constructions previously mentioned 
plus additional syntax to support types and type annotations. These notes use a 
`Programmable Computable Function' (PCF) style syntax and add operations for 
naturals and recursive function definitions.

Specifically these notes formalize the base constructs that make up the 
simply-typed lambda calculus: its syntax, small-step semantics, and typing 
rules. After this a number of properties the language such as progress and 
preservation are stated and proven. The notes may extend the language with 
additional features such as records.

Note: these notes do not present a recommended approach to formalization as 
they eschew a locally nameless representation of terms (via e.g. DeBruijn 
indices). A later section of these notes might look into this. 

\subsection{Imports}

First, we'll need some imports:
\begin{code}
module stlc where 
open import Data.Bool using (Bool; true; false; not)
open import Data.Empty using (⊥; ⊥-elim)
open import Data.List using (List; _∷_; [])
open import Data.Nat using (ℕ; zero; suc)
open import Data.Product using (∃-syntax; _×_)
open import Data.String using (String; _≟_)
open import Data.Unit using (tt)
open import Relation.Nullary using (Dec; yes; no; ¬_)
open import Relation.Nullary.Decidable using (False; toWitnessFalse)
open import Relation.Nullary.Negation using (¬?)
open import Relation.Binary.PropositionalEquality using (_≡_; _≢_; refl)
\end{code}

\section{Term Syntax}

Terms have seven constructs. Three are for the core lambda calculus:
\begin{itemize}
\item variables 
\end{itemize}

\end{document}
