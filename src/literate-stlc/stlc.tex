\documentclass[runningheads]{llncs}
\usepackage[margin=1.85cm, paperwidth=6.1in, paperheight=9in, bottom=2.5cm]{geometry}

\usepackage{mathptmx} % times font instead of cm-modern
\usepackage[T1]{fontenc}

\usepackage{xcolor}
\usepackage{graphicx}
\pagecolor{white}
\definecolor{body}{RGB}{30, 30, 30}
\color{body}

\begin{document}

\title{A Denotational Semantics for STLC}
\author{D. Welch}
\institute{}
\maketitle

\begin{abstract}
Just some notes trying out ``\LaTeX \ literate Agda mode'' as I work through 
this next chapter on denotational semantics.
\end{abstract}

\section{Intro to the Lambda Calculus} 

The lambda calculus, first published by Alonzo Church in 1932 is a minimal calculus
with only three syntactic constructs: variables, abstraction, and (function) 
application. 

The \textit{simply typed lambda calculus} (STLC) is a variant of the lambda calculus
published in 1940 that adds types to the o.g. 1932 untyped lambda calculus.



\subsection{A Subsection Sample}
Please note that the first paragraph of a section or subsection is
not indented. The first paragraph that follows a table, figure,
equation etc. does not need an indent, either.

\end{document}
